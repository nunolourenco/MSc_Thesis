%!TEX root = /Users/nunolourenco/Documents/FCTUC/Mestrado/2010_2011/Thesis/Thesis/thesis.tex
\chapter{DACCO: Discrete Ant Colony Cluster Optimization}
\label{chap:approach}

In this chapter we present DACCO, the ACO algorithm that we used to tackle the problem of Cluster Geometry Optimization. In this dissertation we are interested in developing an effective ACO approach to the problem of cluster geometry optimization. Yet, ACO algorithms were officially proposed for discrete environments and currently there are many variants that are state-of-art methods for different combinatorial optimization problems. Despite a few research efforts \cite{bilchev95, kong06, socha04, tsutsui04}, existing ACO algorithms to continuous domains are somehow incipient, particularly if compared with the most well-known discrete variants such as $\mathcal{MAX}-\mathcal{MIN}$ Ant System or ACS. Hence, in our research we decide to discretized our problem in order to use a discrete variant of ACO. Fig. \ref{fig:dacco_approach} presents and overview of the framework that we proposed.  

\botapic[0.45]{dacco_approach}{Overview of DACCO framework}

The ACO algorithm will build the solutions in the \emph{Discrete-Space}. After, and using \emph{Local Optimization} each solution will be converted to the \emph{Continuous Space} where it will evaluated. Then the \emph{Mapping} will convert the solutions to the discrete space.

This framework poses some interesting research questions that will be addressed:
\begin{enumerate}
	\item Model a real-valued problem in a such a way that it can be solved by a discrete ACO algorithm (\emph{Mapping});
	\item Study the performance of an ACO algorithm in a problem that was not originally discrete;
\end{enumerate}

The variant that we were inclined to use was the $\mathcal{MAX}-\mathcal{MIN}$ Ant System \cite{stutzle00}, since it was one of the most successful approaches of the ACO.  However, it has some drawbacks:
\begin{enumerate}
	\item Definition of the initial values to the pheromone limits;
	\item Readjust the limits every time a new best solution is found; 
	\item The decision in what ant to use to update the pheromones;
\end{enumerate}

To overcome these drawbacks, Blum et al. \cite{blum04}, proposed a variant of the $\mathcal{MAX}-\mathcal{MIN}$ Ant System called Hyper-Cube Framework (HCF). In the HCF the pheromones values are always kept in the interval [0,1]. Hence, we do not have to define an initial limit to the pheromones neither readjust them every time we find a new best solution. Furthermore, the rule to update pheromones in \cite{blum04} uses more than one ant in the updates. They use weights to adjust the relative influence of each ant to the pheromone values. These weights depend on the state of the algorithm.

Taking into account all elements aforementioned, our final choice fell to the HCF, with some modifications, in order to adapt it to our problem. The main modifications are:
\begin{enumerate}
	\item The pheromone update rule does not deposit all the pheromone in one position;
	\item We only use two ants to update the pheromones.
\end{enumerate}

In the next sections we present a high level overview of our entire algorithm. Thereafter we will split it in its main components and we will explain them in more detail.

	
	\pagebreak
	\section{DACCO}
	Here we present a high level description of our ACO algorithm for cluster geometry optimization: \emph{DACCO}. In the following sections, we break it into smaller pieces, and explain them in detail.
	
	\begin{algorithm}
		\caption{DACCO}
		\label{alg:dacco}
		\begin{algorithmic}
		\STATE Construct Search Space
		\STATE Initialize Pheromones
		\WHILE{termination condition not met}
			\STATE Construct Solutions
			\STATE Evaluate Solutions in Continuous Space
			\STATE Convert Solutions to Discrete Space
			\STATE Apply Discrete Local Search
			\STATE Update Pheromone Values
		\ENDWHILE
		\RETURN best individual in the population
		\end{algorithmic}
	\end{algorithm}
	
	\subsection{Construction of Search Space}
	
	In this procedure we discretized the search space, so that we can built a graph where the artificial ants can work.
	The search space is defined by a cube of size $N^{(1/3)}$, where $N$ is the number of atoms, as depicted in Fig. \ref{fig:cube1}. To transform it we decided to divide the cube of Fig. \ref{fig:cube1} into smaller cubes to which we gave the name of \emph{cells}. This cells have to be large enough to contain one atom, but small enough to avoid having to atoms. The final result of this transformation is depicted in Fig. \ref{fig:cube2}.  
	
	\botapic[0.25]{cube1}{Search Space}
	
	\botapic[0.25]{cube2}{Division of the search space in cells}
	\pagebreak
	The Alg. \ref{alg:construct_search_space} gives more details about the construction of the space. It receives the cube size $N^{(1/3)}$, and the cell size \emph{W}.
	
	\begin{algorithm}
		\caption{Construct Search Space}
		\label{alg:construct_search_space}
		\begin{algorithmic}
		\STATE $discretized\_search\_space = \{\}$
		\STATE $max\_coord = N^{(1/3)}$
		\STATE $total\_number\_of\_cells = ceil(max\_coord / W)$
		\FOR{$x = 1 \to total\_number\_of\_cells$}
			\FOR{$y = 1 \to total\_number\_of\_cells$} 
				\FOR{$z = 1 \to total\_number\_of\_cells$}
					\STATE $discretized\_search\_space += (x * W + cell\_center,  y * W + cell\_center, z * W + cell\_center)$
				\ENDFOR
			\ENDFOR
		\ENDFOR
		\RETURN $discretized\_search\_space$
		\end{algorithmic}
	\end{algorithm}

	In the first instruction we begin by defining an empty search space. Then we define the maximum coordinate of our problem. It is important to say that our cube is only defined in the positive side of the xx, yy, zz axis (Fig. \ref{fig:cube1}). After this we calculate the center of each cell in terms of (x,y,z) components and add it to the new search space. 
	The return value is the discretized search space.
	
	The cells that are part of the discretized search space are then used as components $C$ of the graph $G$ that the ants use to build solutions. The edges $E$ are arcs connecting all the components of the graph. In the following sections we will use the name \emph{cells} instead of components since in our approach they are the same thing.
	
	\subsection{Initialize Pheromones}
	Before we start the optimization process, we need to initialize the pheromone matrix. Following \cite{blum04}, we choose the value of $0.5$ as the initial pheromone values.

	\subsection{Construction of Solutions}
	The construction of ant solutions is achieved by the method Construct Solutions. An ant solution corresponds to a complete atomic cluster. The algorithm starts by defining a start position for the ant, thus we have to place the ant in a cell of our search space. Then the ant calculates which are the neighbors of it, and chooses, in a probabilistic way, a new cell to move. After it knows which cell is the next, it places the atom in the center of the current cell, add it to solution, and moves to next one. This process is repeated until each ant has a cluster with all the atoms in place. The general algorithm that defines this construction process is detailed in Alg. \ref{alg:construct_solutions}.
	
	\begin{algorithm}
		\caption{Construct Solutions}
		\label{alg:construct_solutions}
		\begin{algorithmic}
		\STATE Given an ant \bf{do}:
		\STATE $placed\_elements = 1$
		\WHILE{$placed\_elements < N$}
			\STATE $neighbors = find\_feasible\_neighborhood(ant)$
			\STATE $next\_cell = find\_next\_cell(ant, neighbors, pheromone\_matrix)$
			\STATE $ant.solution[placed\_elements] = Atom(ant.current\_cell)$
			\STATE $ant.current\_cell = next\_cell$
			\STATE $placed\_elements = placed\_elements + 1$			
		\ENDWHILE
		\end{algorithmic}
	\end{algorithm}
	
	In the end of Alg. \ref{alg:construct_solutions} an ant will have built a complete atomic cluster.
	
	We now focus our attention in some of the procedures used:\\ the $find\_feasible\_neighborhood()$ and $find\_next\_cell()$. 
	
		\subsubsection*{Find Feasible Neighborhood}
		This procedure determines the cells that are available in the neighbor of the current position of an ant. To find this neighbors, we used some different techniques:
		\begin{enumerate}
			\item Moore Neighborhood
			\item Convex Hull
			\item Full Moore Neighborhood
		\end{enumerate}
	
		\paragraph*{Moore Neighborhood}
			This neighborhood is defined, in $\mathbb{R}^3$, by the cube that is centered in a the current cell (x0, y0, z0). The Moore neighborhood of range r can be defined by the set M of all points (x, y, z) that verify the following condition:
			\begin{equation}
				M= \{(x,y):|x-x0| \leq r \wedge |y-y0| \leq r \wedge |z-z0| \leq r\}
			\end{equation}
			where $r \geq 0$. However, in our approach the values of r are only in the range r > 0, once we need to have at least one cell that is different from (x0, y0, z0). Fig. \ref{fig:moore_neighborhood} depicts the a Moore neighborhood with $r = 1$. We use a 2D representation for simplicity. 
			
			\botapic[0.50]{moore_neighborhood}{The grey cells represent the final M set, with $r = 1$}
			
			The final M set will then be used to determine which cell will be the next.
			
			\paragraph*{Convex Hull}
			With this technique we wanted to use more information about the problem, in the choice of the neighbors. Since the Morse potential takes in account the distance of all pair of atoms that composes the aggregate, we thought that we should use the information of the partially constructed solution to help choose the neighborhood. One of the first ideas that came into mind was the following: build a convex hull, with the atoms that are in the partial solution, and then determine all the cells that are at distance one from the segments defined by the points of the convex hull. The process is detailed in Alg.4.
			\begin{algorithm}
				\caption{Convex Hull}
				\label{alg:convex_hull}
				\begin{algorithmic}
				\STATE Given a partial solution of an ant \bf{do}:
				\STATE $M = \{\}$
				\STATE $neighbors = \{\}$
				\STATE $ch = find\_convex\_hull(partial\_solution)$
				\STATE $ch\_size = length(ch)$
				\FOR{$i = 1 \to ch\_size$}
					\FORALL{cells $CL$ of $discretized\_search\_space$}
						\IF{$(distance(CL, segment(ch[i-i], ch[i])) = 1$}
							\STATE $neighbors += CL$
						\ENDIF
					\ENDFOR
				\ENDFOR
				\STATE $M = remove\_repeated\_cells(neighbors)$
				\RETURN $M$
				\end{algorithmic}
			\end{algorithm}
			
			The $remove\_repeated\_cell()$ procedure removes the repeated cells that are in the M set, since one cell can be in more the one atom neighborhood.
			The M set will then be used to determine which cell should be added to the solution.

			This technique uses the information of all the atoms that had already been placed, but it demands a lot of computational resources. Every time we place a new atom we have to calculate the Convex Hull of the current partial solution. This, together with the additional time on the calculus of the distance of all cells to the segments, made us look for other alternatives.
			
			\paragraph*{Full Moore Neighborhood}
			This technique was another attempt to use information about the partial constructed cluster.  In this technique, we iterate by all atoms that are in the partial solution, and we look for the neighbors of them. The neighborhood that we use is the Moore neighborhood with the same $r$ for all the atoms. The general idea is depicted in Fig. \ref{fig:all_neighbors}. We use a 2D representation for simplicity. The following algorithm gives more details about this technique:
		
			\begin{algorithm}
				\caption{Full Moore Neighborhood}
				\label{alg:all_atom_neighbors}
				\begin{algorithmic}
				\STATE Given a partial solution of an ant \bf{do}:
				\STATE $M = \{\}$
				\STATE $neighbors = \{\}$
				\FOR{$i = 1 \to ch\_size$}
					\FORALL{atoms $AT$ of $partial\ solution$}
						\STATE $neighbors += Moore\_Neighborhood(AT)$
					\ENDFOR
				\ENDFOR
				\STATE $M = remove\_repeated\_cells(neighbors)$
				\RETURN $M$
				\end{algorithmic}
			\end{algorithm}
		
		
		\botapic[0.50]{all_neighbors}{The grey cells represent the final M set, with $r = 1$}	
		
		\subsubsection*{Find Next Cell}	
		This procedure receives the feasible neighborhood and determines the next cell to be part of the solution. This means that, an ant, located in a cell $i$ should choose a new cell $j$ to move to. This choice is based on the pheromone value of cell $j$, $\tau_{j}^{\alpha}$ and the heuristic information $\eta_{j}^\gamma$. It is important to say that we do not use the heuristic component of the choice function. The heuristic information is applied when we are building the feasible neighborhood. In the case of our problem we want to place atoms that are close to each other. Thus we can discard the cells that are too far away from the current one. 
		Hence, the choice function depends only of the pheromone value of cell $j$:
		\begin{equation}
			\label{eq:prob_rule}
			p_j = \frac{[\tau_j]^\alpha} {\sum_{l \in M} [\tau_l]^\alpha}
		\end{equation}
		
		Providing all values of the rule function are stored in $V$, the iterative cell selection is determined by a roulette wheel algorithm.: each value of the $V$ determines a slice on a circular roulette wheel. Next, the wheel is spun and the cell to which the marker points is chosen as the next cell for the ant. 
		In Alg. \ref{alg:find_next_cell} we detail the find next cell procedure, and in Alg. \ref{alg:roulette_wheel} we detail the roulette wheel selection method. 
		
			\begin{algorithm}
				\caption{Find Next Cell}
				\label{alg:find_next_cell}
				\begin{algorithmic}
				\STATE Given a set $M$ of candidate cells \bf{do}:
				\FORALL{cells in $M$}
						\STATE $V =$ Apple Eq. \ref{eq:prob_rule} the probability of each cell
				\ENDFOR
				\RETURN $roullete\_wheel(V)$
				\end{algorithmic}
			\end{algorithm}
			
			\begin{algorithm}
				\caption{Roulette wheel}
				\label{alg:roulette_wheel}
				\begin{algorithmic}
				\STATE Given a set $V$ of probabilities \bf{do}:
				\STATE $index = 0$
				\STATE $total = V[0]$
				\STATE pick a random value $r$ uniformly from $[0,1]$
				\WHILE{$total < r$}
					\STATE $index = index + 1$
					\STATE $total = total + V[index]$		
				\ENDWHILE
				\RETURN $index$
				\end{algorithmic}
			\end{algorithm}
			
			After finding the next cell, the ant will place the atom in its center, and then repeat all this process until it has a complete solution, that is, all the atoms have been placed.
			
			\subsection{Evaluate Solution in Continuous Space}
			
			The evaluation of the solutions, which were built by the ants, is made in this procedure. Since we have the atoms in the center of the cell, which are represented by $(x,y,z)$, where x, y, z $\in \mathbb{R}$, we do not have to make any additional computation to pass from the discretized space to the continuous space.
			
			The evaluation of the solutions proceeds in two steps. First, the Broyden-Fletcher-Goldfarb-Shannon (L-BFGS) quasi-Newton method \cite{liu89} move the solution to the nearest local optimum. Then equation \ref{eq:morse_potential} is used to determine the potential energy of the resulting cluster.

			L-BFGS is an efficient local optimization method that combines the modest storage and computational requirements of conjugate gradient methods with the super linear convergence exhibited by full memory quasi-Newton strategies. This local optimization algorithm is usually adopted by hybrid approaches for cluster geometry optimization problems \cite{grosso07, johnston03, xico09}.
						
			\subsection{Convert Solutions to Discrete Space}
			In this procedure we convert the solutions, which were returned by Evaluate Solutions in Continuous Space procedure, to the discrete space. This is very important, because, as we referred, the solutions returned after their evaluation are different from the ones that were given as parameters.
			To convert the cluster to the discretized space, we apply the following algorithm: given a certain atom, we calculate the distance of this atom to all centers of the cells in the space, and the cell that is closer the atom, will be the one that will hold it. This process is detailed in Alg. \ref{alg:convertion_discrete}.
			
			\begin{algorithm}
				\caption{Convert Solution to Discrete Space}
				\label{alg:convertion_discrete}
				\begin{algorithmic}
				\STATE Given a solution $CS$ in the continuous space \bf{do}:
				\STATE $discrete\_solution = \{\}$
				\FORALL{atoms $AT$ of $CS$}
						\STATE $distances = \{\}$
						\FORALL{cels $CL$ of $discretized\ space$}
								\STATE $distances[AT][CL] = distance(AT, CL)$
						\ENDFOR
						\STATE $discrete\_solution += min\_dist\_cell(AT, distances)$
				\ENDFOR
				\RETURN $discrete\_solution$
				\end{algorithmic}
			\end{algorithm}
			
			The $min\_dist\_cell()$ procedure receives an array distance of atom to cells, and returns the cells that is closer to the atom in question.
			
			After this, we move all the cluster to the origin of the axis. This allows us to have the information concentrated in a place of the search space. 
			
			\subsection{Apply Discrete Local Search}
			This procedure is responsible for applying some perturbations in the solutions that were built by the ants, in an attempt to improve the search results.

			After the solutions have been converted to the discrete space, we look for the atom that has the worst contribution to the Morse potential, and we move it to a random cell.  After this, we evaluate the solution again, and if the result is a better solution, we keep it. This process is repeated for a given number of tries. The Alg. \ref{alg:discrete_local_search} gives a description of the process:
	
			\begin{algorithm}
				\caption{Apply Discrete Local Search}
				\label{alg:discrete_local_search}
				\begin{algorithmic}
				\STATE Given an ant solution AS \bf{do}:
				\STATE $i = 0$
				\STATE $best\_solution = AS$
				\STATE $current\_solution = AS$
				\WHILE{$i < local\_search\_iterations$}
					\STATE $worst\_atom\_cell = find\_atom\_worst\_contribution(current\_solution)$
					\STATE $new\_position = random\_cell()$
					\STATE $current\_solution[worst\_atom\_cell] = new\_position$
					\STATE $evaluate(current\_solution)$
					\IF{$is\_better(current\_solution, best\_solution)$}
						\STATE $best\_solution = current\_solution$
					\ELSE
						\STATE $current\_solution = best\_solution$
					\ENDIF
				\ENDWHILE
				
				\RETURN $best\_solution$
				\end{algorithmic}
			\end{algorithm}		
			
			The $local\_search\_iterations$ are the number of iterations that we make to try improve the solutions.

			The $random\_cell()$ procedure returns a random cell that belongs to the search space, and that is not already occupied.

			The $is\_better()$ procedure checks if the first solution is better than the second one.
			\subsection{Update Pheromone Values}
			This procedure is responsible for the update of the pheromones. 

			We start by decrease the current values of the pheromones, by a certain percentage, simulating the pheromone evaporation in the nature.

			After, to deposit pheromones we follow the rule presented in \cite{blum04} with some minor differences. In the cited approach, they used 3 ants to update the trails:
			\begin{enumerate}
				\item \emph{Iteration-best ant} - which corresponds to the best ant in the current iteration of the algorithm;
				\item \emph{Restart-best ant} - which corresponds to the best ant found since the restart of the algorithm
				\item \emph{Global-best ant} - which corresponds to the best solution found since the start of the algorithm
			\end{enumerate}
			
			However, in our approach we do not have a restart mechanism, thus we do not need to use the restart-best ant. 

			As was pointed out by \cite{blum04}, finding a schedule for the usage of the ants can be a difficult task, and require a lot of experimentation. However in the HCF, we do not have this problem, because both iteration-best and global-best ants are allowed to deposit pheromones. The influence of each ant is measure by the weights that we assign to each one. The final rule to update pheromones is depicted by the following:
			\begin{equation}
				\tau = \tau + \omega
			\end{equation}
			
			where,
			
			\begin{equation}
				\omega = w_{ib} * F_{ib} + w_{gb} * F_{gb}
			\end{equation}
			where, $w_{ib}$ is the weight of the iteration-best ant, $F_{ib}$ is the quality of the iteration-best ant, $w_{gb}$ is the weight of the global-best ant, $F_{gb}$ is the quality of the global best ant, and $w_{ib} + w_{gb} = 1$.
			
			After the update is applied, pheromone values that exceed $\tau_{max}$ are set back to $\tau_{max}$. The similar process is applied to $\tau_{min}$.

			The final important aspect of the pheromone update rule is that the pheromones are not deposit in only cell. Instead, we only deposit a percentage $p$ in the main cell, and we propagate a percentage $(1-p)$ to the adjacent cells (Fig. \ref{fig:propagation}). This propagation is uniformly distributed among all the adjacent cells. This means that the value that is given to each one of the adjacent cells is $(1-p) / (number\ of\ adjacent\ cells)$. Such aspect is important to maintain a homogeneous pheromone matrix.
			
		\botapic[0.7]{propagation}{Pheromone Propagation}	
			
			
			
			
			