%!TEX root = /Users/nunolourenco/Documents/FCTUC/Mestrado/2010_2011/Thesis/Thesis/thesis.tex
\chapter{Implementation}

Since this dissertation is part of the Masters Degree in Informatics Engineering, it is import to refer some technical aspects about the implementation. In this chapter we give a brief description of the technical choices made, and we present a quick overview of the entire system.

\section{Technical Choices}

To implement the first versions of the system we decide to use the Python language. One of the reasons for this is related with the simplicity of the language, and easiness in making changes. The other reason is related with the L-BFGS method, which is written in Fortran. With python it is easy to connect modules made in Fortran with our code.

After we have a stable version, we decided to write it in C. The reason for this was the performance of the python language. 

The C language, despite being very fast, has some programming overhead, when compared, for instance, with python. During the conversion of the python code to C, some bugs appeared, as one should expect. They were solved using the \emph{gdb} debugging system.

We used \emph{Git} as revision control system. As online repository our choice was the \emph{Github} platform, with private repositories.

\section{Architecture Overview}
We tried to keep our architecture simple and modular. In Fig.\ref{fig:class_diagram} we present a class diagram, which in our opinion gives a better insight in how the system is organized.

\botapic[0.7]{class_diagram}{CGACO Class Diagram}
\pagebreak
\textbf{CGACO} – Main class that runs the ACO algorithm for a certain number of populations.

\textbf{Population} – Class that simulates the ant colonies that appears in nature. It has a certain number of individuals, with common properties.

\textbf{Ant} – Class that simulates the real ant. It is responsible for the individual behavior of each ant. This behavior includes, for instance, decide directions to go while building solutions.

\textbf{Utils} – Class that has a set of procedures, that help all the other classes accomplish their individual tasks.

\textbf{Fitness Function} – Class that is responsible for evaluate the quality of the individuals that belong to our system.

