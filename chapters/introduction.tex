%!TEX root = /Users/nunolourenco/Documents/FCTUC/Mestrado/2010_2011/Thesis/Thesis/thesis.tex
\chapter{Introduction}
Challenging optimization problems occur in many science and engineering fields. In the Chemistry field, the interest of effective optimization algorithms arises from the problem of finding the lowest energy configuration of an atomic/molecular cluster. For example, it seems that the native structure of a protein is related with the lowest energy configuration of atoms that composes it\cite{wales97}. If we could find this structure effectively and reliably from the amino-acid sequence, this knowledge would provide new insights into the nature of protein folding. These insights would be helpful, for example, for the development of drugs by pharmaceutical companies.

The problem of finding the lowest energy configuration of an atomic/molecular cluster is usually designated by \emph{Cluster Geometry Optimization}. This problem studies how should a set of atoms/molecules be placed in a 3-dimensional space, in order to get the minimal configuration energy. During the last years, a lot of effort has been made to develop effective algorithms for this type of problem.

In this report we aim to present the current state of cluster geometry optimization, and to propose a new Swarm Intelligence approach to solve this problem: the Ant Colony Optimization.

The remainder of this document is the following: in the next chapter we introduce the problem of cluster geometry optimization. In Chapter \ref{chap:opt_alg} we describe some optimization algorithms, and highlight the main achievements in what concerns their application to cluster geometry optimization. In Chapter  we detail the goals of our research and present the approach that we intend to follow in the development of ant colony optimization algorithms for cluster geometry optimization. Chapter \ref{chap:curr_work} presents some results of current algorithms that are used in cluster geometry optimization. Finally, Chapter \ref{chap:conclusions} gathers the main conclusions.



