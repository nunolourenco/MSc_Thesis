%!TEX root = /Users/nunolourenco/Documents/FCTUC/Mestrado/2_ano/Thesis/Thesis/thesis.tex
\chapter{Conclusions}
\label{chap:conclusions}

With this work we aim to build an ACO algorithm for the cluster geometry problem. This problem is difficult to solve, and the ACO framework has shown very good results in solving hard problems. 

We will follow an hybrid approach: the solution construction occurs in a discrete space, while the quality evaluation occurs in the continuous space. This we will be made in two steps: first a local search algorithm will take the current solution to the closest local optimum; second it determines the current quality of the solution.

To the best of our knowledge this is the first approach that uses ACO algorithms for this type of optimization. We want to evaluate the capacity of the ACO in the discovering of good solutions for the cluster geometry problem.

In order to appreciate the results we will compare our results with state-of-art approaches for cluster geometry optimization.
