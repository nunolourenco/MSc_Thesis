%!TEX root = /Users/nunolourenco/Documents/FCTUC/Mestrado/2_ano/Thesis/Thesis/thesis.tex
\chapter{Research Objectives}
\label{chap:research_obj}
%Aplicar ACO ao CGO


The main goal of our research is to apply ACO algorithms to cluster geometry optimization problem. More specifically, ACO methods will seek for low energy structures of different instances of short-ranged Morse clusters. 

ACO frameworks for continuous domains are less effective then variants for discrete spaces. Our approach will then focus on discrete ACO algorithms. Mapping to the continuous defined by cluster geometry optimization will be accomplished by a local search procedure. In Section \ref{sec:approch_method} we detail the process. We choose this method because:
		\begin{enumerate}
			\item discrete ACO algorithms are particularly effective and have been successfully applied to real-life problems \cite{acobook}, \cite{Loiola2007657}, \cite{ hoos03}, \cite{StutzleTSP99};
			\item all state-of-art approaches for cluster geometry optimization are based on a hybrid combination of global optimization methods and a gradient driven local search procedure. In our approach, local search is also essential to allow an accurate approximation of the Morse potential;
			\item to the best of our knowledge it has never been applied to cluster geometry optimization.
		\end{enumerate}

	Our proposal will be compared with some of the current methods of cluster geometry optimization. The comparison will be made using clusters with a number of atoms between 30 and 50. Additionally we will perform a statistical study to assess the quality of the results.
	\pagebreak
\section{Approach}
	%Explicar muito por alto as ideias que tenho e como as vou aplicar
	\label{sec:approch_method}
	After a careful study of the ACO framework and the Cluster Geometry Optimization problem, we are inclined to follow an approach that divides the 3-dimensional space in a set of cells giving rise to a grid. The size of each cell should be large enough in order to an atom fit inside, and small enough for not letting two atoms inside. Then, and as specified in Section \ref{subsec:swarm_intelligence} we will use these cells to build a complete graph $G$, which will be the ACO search space. This process is depicted in Fig. ~\ref{fig:approach_scheme}. For visualization proposes we use a scheme in the 2D space rather than in the 3D.
	
	\botapic[0.8]{approach_scheme}{Search space divided into a 2D grid}
	
	 Next, we will use an ACO method to build a set of possible solutions. The building of solutions is made as follows: starting from a cell in the grid, an ant starts to put an atom in that cell; then, it moves to the next cell, using one of the rules described Section \ref{subsec:swarm_intelligence}, and puts another atom in that cell. This process is repeated until we have a solution that corresponds to a \emph{N}-atom clusters, with all atoms distributed. The Fig. \ref{fig:solutionbuilding} shows the procedure.
	\botapic[0.8]{solutionbuilding}{Solution Construction}
	
	 These solutions will then be sent to a gradient local search algorithm that will use the first derivative of Eq. \ref{eq:morse_potential} to guide them to the nearest local optimum. For more details on the local search algorithm please refer to Liu et al. \cite{liu89}. After the local search process, we will be in the discrete space again. We check the new positions of the atoms in the mesh and update the value of the pheromone in each position of the grid. Hopefully, and using the information given by the pheromones, in the later stages of the optimization the ants will build only clusters that have good values of the morse potential.
	
\section{Work Plan}
The implementation of the approach described above will occur during the second semester, corresponding to the second part of this work. The process will be divided by phases, defined as follows:
	\begin{enumerate}
		\item Implement a first approach to cluster geometry optimization using ACO.
		\item Evaluate the results obtained by the first prototype.
		\item Searching of alternatives that help to enhance the effectiveness of the approach.
		\item Assessment of the best setting for the implement model.
		\item Run experiences for the statistical study with the best setting found.
		\item Statistical study, where we will compare the proposed method with two other methods used for cluster geometry optimization.
		\item Writing of the final report.
	\end{enumerate}
	
	For more details confer the gantt graph in Fig. \ref{fig:WorkPlan}.
	
	%\begin{landscape}
	%	\centering
	%	\botapic[0.6]{WorkPlan}{Work Plan}
	%\end{landscape}
	
	
	
	

