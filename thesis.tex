\documentclass[12pt,a4paper]{report}


\usepackage{mystyle}
\usepackage{pdflscape}
\usepackage{rotating}

\setcounter{tocdepth}{3}
\setcounter{secnumdepth}{3}

\begin{document}
	\pagenumbering{roman}
	\input{"chapters/title"}
	
	\begin{abstract}
		The problem of cluster geometry optimization is relevant for many areas from protein structure prediction to the field of nanotechnology. A cluster is an aggregate of interacting atoms or molecules and it can hold a few or even millions of elements. Finding the organization for the \mbox{atoms/molecules} that has the lowest potential energy is an NP-hard problem. In this dissertation we propose an approach based on Swarm Intelligence algorithms. Specifically, we describe the application of an algorithm based on Ant Colony Optimization to the cluster geometry optimization problem.
		Results are promising, since the the proposed approach is able to discover almost all the best-known solution for short-ranged Morse clusters, between 30 and 50 atoms. A comparative analysis with some state-of-art algorithms is presented and it shows that the approach can be as effective as the algorithms used for comparison. Moreover we perform an analysis in some of the components of the approach to observe the influence that each has on the algorithm.
		
		
		
		
		\textbf{Keywords:} Cluster geometry optimization, Morse Cluster, Swarm Intelligence, Ant Colony Optimization
	\end{abstract}
	\tableofcontents
	\listoffigures
	\pagebreak
	\pagenumbering{arabic}
	\input{"chapters/introduction"}
	\input{"chapters/atomic_clusters"}
	\input{"chapters/optimization_of_clusters"}
	\input{"chapters/dacco"}
	\input{"chapters/implementation"}
	\input{"chapters/results"}
	\input{"chapters/conclusions"}
	\bibliographystyle{plain}
	\addcontentsline{toc}{chapter}{Bibliography}
	\bibliography{refs}
	
\end{document}

